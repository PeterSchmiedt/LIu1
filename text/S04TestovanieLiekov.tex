\section{Testovanie jednotlivých liekov}

Každá podskupina (so svojím vlastným ID) je testovaná na účinnosť lieku 1 alebo 2. Resp. ktorý liek je účinnejši.

Budeme testovať nasledujúcu hypotézu: ``Medzi liekom 1 a 2 nie je žiadny rozdiel účinnosti.'' Hypotézu budeme testovať za pomoci štatistického testu dobrej zhody (chí-kvadrát test) \cite{bib:wiki-chi-squared} \cite{bib:biomedicina-statistika}. Výsledok týchto testov môžme vidieť v tabuľke v prílohe \ref{app:ucinnost-liekov}.

V niektorých podskupinách nie je potvrdená účinnosť ani jedného z liekov a to už z dôvodu malej vzorky (menej ako 25 pacientov) alebo z neúčinnosti oboch liekov.


\subsection{Vplyv liekov na pridružné choroby}

Testovali sme aj vplyv liekov na pridružené sekundárne choroby.

\begin{table}[h!]
\centering
\begin{tabular}{l|ll}
\hline
                          & \textbf{Liek 1} & \textbf{Liek 2} \\ \hline
\textbf{1. sek. choroba+} &                 &                 \\ \hline
\textbf{1. sek choroba--} &                 &                 \\ \hline
\textbf{2. sek choroba+}  &                 &                 \\ \hline
\textbf{2. sek choroba--} &                 &                 \\ \hline
\end{tabular}
\caption{Vplyv liekov na pridružené ochorenia}
\label{tab:sekundarne-ochorenia}
\end{table}