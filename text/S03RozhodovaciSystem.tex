\section{Rozhodovací systém}

Prieniky jednotlivých kategórií si budeme značiť systémom ID, kde ID bude 4-ciferné číslo a každá cifra znamená nejakú konkrétnu kategóriu. Prvá cifra udáva vek, druhá cifra udáva MAP, tretia BMI a štvrtá cifra nám udáva prítomnosť inej medikácie počas liečby.

\begin{table}[!h]
\centering
\begin{tabular}{cc|
>{\columncolor[HTML]{38A4DA}}c 
>{\columncolor[HTML]{38A4DA}}c |
>{\columncolor[HTML]{F2BD35}}c 
>{\columncolor[HTML]{F2BD35}}c |
>{\columncolor[HTML]{32CB00}}c 
>{\columncolor[HTML]{32CB00}}c }
\hline
\textbf{Vek} & \textbf{}   & {\color[HTML]{333333} \textbf{MAP}} & {\color[HTML]{333333} \textbf{}}    & \textbf{BMI} & \textbf{} & \textbf{Prítomnosť inej medikácie} & \textbf{} \\ \hline
1000         & \textless30 & {\color[HTML]{333333} 100}          & {\color[HTML]{333333} \textless70}  & 10           & Podvýživa & Prítomná                           & 1         \\ \hline
2000         & 30-39       & {\color[HTML]{333333} 200}          & {\color[HTML]{333333} 70-92}        & 20           & Normál    & Neprítomná                         & 0         \\ \hline
3000         & 40-49       & {\color[HTML]{333333} 300}          & {\color[HTML]{333333} 93-105}       & 30           & Nadváha   &                                    &           \\ \hline
4000         & 50-59       & {\color[HTML]{333333} 400}          & {\color[HTML]{333333} 106-119}      & 40           & Obezita   &                                    &           \\ \hline
5000         & 60\textless & {\color[HTML]{333333} 500}          & {\color[HTML]{333333} 120\textless} &              &           &                                    &           \\ \hline
\end{tabular}
\caption{Rozhodovací systém}
\label{tab:rozhodovaci-system}
\end{table}

Napríklad: ID-2321 znamená že ide o pacientov vo veku 30-39 rokov, s MAP v rozsahu 93-105mmHg, BMI normálne a s prítomnosťou inej medikácie.